\documentclass{article}

\usepackage{amsmath}
\usepackage[hidelinks]{hyperref}
\usepackage{booktabs}
\usepackage{threeparttable}

\title{Search}
\author{Tristan Slater}

\begin{document}
    \maketitle

    \tableofcontents

    \section{Defining Search Problems}

    \begin{enumerate}
        \item Initial state
        \item Successor
        \item Goal test
        \item Step cost
    \end{enumerate}

    \subsection{Successor Function}

    Tells us where we can go from a state: \begin{equation}
        S(s_0) = \{(a_1, s_1), ..., (a_n, s_n)\}
    \end{equation}

    When actions are inferrable from states: \begin{equation}
        S(s_0) = \{s_1, ..., s_n\}
    \end{equation}

    \subsection{Goal Test Function}

    Tells when solution has been reached. \textbf{Explicit} defines which state is a goal. \textbf{Implicit} is some set of conditions that could happen at various states under various conditions.

    \subsection{Step Cost Function}

    Tells how much a single step costs: \begin{equation}
        c(s_1, a, s_2) \geq 0
    \end{equation}

    \subsection{Solution}

    Any path from initial state to goal state: $$\{a_1, ..., a_n\}$$
    
    When actions are inferrable from states: $$\{s_1, ..., s_n\}$$

    \section{Basic Search}

    A \textbf{node} is an abstract data structure that contains information about a state, parent node, action, path cost, and depth (optional).

    The \textbf{frontier} is the collection of nodes that have been \textit{generated}, but \textit{not expanded}.

    \begin{table}[h]
        \caption{Comparison of Basic Search Strategies}
        \label{tab:basic-search-comparison}
        \centering
        \begin{threeparttable}
            \begin{tabular}{lccc}
                \toprule
                & BFS & UCS & DFS \\
                \midrule
                Node ADS & Queue & Min queue\tnote{3} & Stack \\
                Nodes to Expand First & Shallowest & Cheapest & Deepest \\
                Complete? & Yes\tnote{1} & Yes\tnote{4} & No \\
                Optimal? & Yes\tnote{2} & Yes & No \\
                Time Complexity & $O(b^{d + 1})$ & $O(b^{\lceil C^*/\varepsilon\rceil})$\tnote{5} & $O(b^m)$ \\
                Space Complexity & $O(b^{d + 1})$ & $O(b^{\lceil C^*/\varepsilon\rceil})$ & $O(bm)$ \\
                \bottomrule
            \end{tabular}
            \begin{tablenotes}
                \item[1] If $b$ finite
                \item[2] If step costs equal
                \item[3] Path cost
                \item[4] If step cost $>0$
                \item[5] $C^*$ cost of solution
            \end{tablenotes}
        \end{threeparttable}
    \end{table}

    \textbf{DLS} is exactly like DFS except with the depth limited.

    \textbf{IDS} is exactly like DLS except it tries all depths starting at 0.
    
    IDS is \textit{complete}.
    
    IDS is \textit{optimal} if the step cost is 1.
    
    \section{Best-First Search}

    Like UCS, but uses a \textbf{evaluation function} $f(n)$ to pick the \textit{best} node to expand first.

    \begin{table}[h]
        \caption{Comparison of Best-First Search Strategies}
        \label{tab:informed-search-comparison}
        \centering
        \begin{threeparttable}
            \begin{tabular}{lcc}
                \toprule
                & Greedy & A\textsuperscript{*} \\
                \midrule
                Evaluation Function & $h(n)$ & $g(n) + h(n)$ \\
                Complete? & No & Yes \\
                Optimal? & No & Yes\tnote{7} \\
                Time Complexity & $O(b^m)$ & - \\
                Space Complexity & $O(b^m)$ & - \\
                \bottomrule
            \end{tabular}
            \begin{tablenotes}
                \item[7] For tree search, if $h(n)$ is admissible
            \end{tablenotes}
        \end{threeparttable}
    \end{table}

    \subsection{A\texorpdfstring{\textsuperscript{*}}{ Star}}

    $g(n)$: Cost so far to reach $n$.

    \subsection{Heuristics}

    \subsubsection{Admissibility}

    If heuristic is at most the true cost for every node: $$h(n) \leq c(n) \text{ for all } n$$

    Time/space complexity depends on heuristics.
    
    \subsubsection{Dominance}

    $h_2(n)$ \textit{dominates} $h_1(n)$ if $h_2(n) \geq h_1(n)$ for all $n$.

    If $h_2(n)$ is admissible, then $h_1(n)$ is admissible.
\end{document}
